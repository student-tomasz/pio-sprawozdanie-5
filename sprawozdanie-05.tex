\documentclass[10pt,a4paper]{article}
\usepackage[a4paper]{geometry}

\usepackage{polski}
\usepackage{xltxtra}
\usepackage{indentfirst}
\usepackage{relsize}
\usepackage{fancyvrb}
\usepackage[pdfborder={0 0 0}]{hyperref}

%% tweak fonts
\defaultfontfeatures{Mapping=tex-text}
\setromanfont{Charis SIL}
\setsansfont[Scale=MatchLowercase]{Helvetica Neue}
\setmonofont[Scale=MatchLowercase]{Menlo}
\linespread{1.25}

%% define custom commands and environments
\DefineVerbatimEnvironment%
  {SmallVerbatim}%
  {Verbatim}{fontsize=\relsize{-0.5},numbers=left,numbersep=-10pt,frame=lines,tabsize=4}

\newcommand{\f}[1]{\texttt{#1}}
\newcommand{\s}[1]{\textsf{#1}}

\begin{document}

%%fakesection{Tytuł}
\title{
  Sprawozdanie nr~5 z~laboratorium\\Podstaw Inżynierii Oprogramowania
}
\author{
  Grzegorz Bartkowiak\\
  Tomasz Cudziło\\
  Mateusz Ochtera\\
  Gustaw Wypych\\
  \\
  \textsc{PW EE Informatyka}\\[10pt]
}
\date{\today}
\maketitle

\section{Weryfikacja systemu aukcyjnego}

Dokument opisuje w~częściach proces weryfikacji systemu aukcyjnego. Zawiera
przykładowe testy modułów systemu, wydajnościowe i~akceptacyjne.

Każdy scenariusz testowy musi posiadać opis wykonywanych czynności, opis danych
wejściowych oraz kryteria wyznaczające powodzenie testu.

\section{Testy modułów}

\section{Testy wydajnościowe}

\section{Testy akceptacyjne}

\subsection{Udział w~aukcji}
\begin{description}
  \item[Opis testu] \hfill \\
    Test ma zweryfikować czy udział w~aukcji jest zrozumiały, wygodny oraz
    odpowiednio responsywny do przeprowadzania aukcji w~czasie rzeczywistym.
    Testerzy używają systemu, wypełnionego fikcyjnymi aukcjami w~różnych
    stanach i~biorą udział w~aktywnych.
  \item[Przygotowanie do testu] \hfill \\
    Wygenerowanie fikcyjnych aukcji.
  \item[Kryteria oceny] \hfill \\
    Wszystkie powiadomienia o~aukcjach są dostarczane tj. do właściciela
    przedmiotu, do kupującego, do pracowników, do magazynu.\\
    Klienci są w~stanie brać udział w~aukcjach i~nie uważają procesu za
    trudny.\\
    Przebieg aukcji i~jej strona są odświeżane automatycznie bez, lub
    z~opóźnieniem na tyle małym, że nie wpływa bezpośrednio na przebieg aukcji.
\end{description}

\end{document}

