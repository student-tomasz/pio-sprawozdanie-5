\documentclass[10pt,a4paper]{article}
\usepackage[a4paper]{geometry}

\usepackage{polski}
\usepackage{xltxtra}
\usepackage{indentfirst}
\usepackage{relsize}
\usepackage{fancyvrb}
\usepackage[pdfborder={0 0 0}]{hyperref}

%% tweak fonts
\defaultfontfeatures{Mapping=tex-text}
\setromanfont{Charis SIL}
\setsansfont[Scale=MatchLowercase]{Helvetica Neue}
\setmonofont[Scale=MatchLowercase]{Menlo}
\linespread{1.25}

%% define custom commands and environments
\DefineVerbatimEnvironment%
  {SmallVerbatim}%
  {Verbatim}{fontsize=\relsize{-0.5},numbers=left,numbersep=-10pt,frame=lines,tabsize=4}

\newcommand{\f}[1]{\texttt{#1}}
\newcommand{\s}[1]{\textsf{#1}}

\begin{document}

%%fakesection{Tytuł}
\title{
  Sprawozdanie nr~5 z~laboratorium\\Podstaw Inżynierii Oprogramowania
}
\author{
  Grzegorz Bartkowiak\\
  Tomasz Cudziło\\
  Mateusz Ochtera\\
  Gustaw Wypych\\
  \\
  \textsc{PW EE Informatyka}\\[10pt]
}
\date{\today}
\maketitle

\section{Weryfikacja systemu aukcyjnego}

Dokument opisuje w~częściach proces weryfikacji systemu aukcyjnego. Zawiera
przykładowe testy modułów systemu, wydajnościowe i~akceptacyjne.

Każdy scenariusz testowy musi posiadać opis wykonywanych czynności, opis danych
wejściowych oraz kryteria wyznaczające powodzenie testu.

\section{Testy jednostokowe}

\subsection{Moduł rejestracji użytkowników}

\subsubsection{Test na poprawnych danych}
\begin{description}
  \item[Opis testu] \hfill \\
    Test sprawdza poprawność obiektu typu \f{Klient} tworzonego przez metodę
    przyjmującą dane klienta.
  \item[Dane wejściowe] \hfill \\
    Poprawne, fikcyjne dane klienta.
  \item[Kryteria oceny] \hfill \\
    100\% zgodność danych w~obiekcie z~danymi wejściowymi.
\end{description}

\subsubsection{Testy na nieprawidłowych danych}
Testy jednostkowe na nieprawidłowych danych mają identyczny przebieg. Każdy
test posiada dane wejściowe posiadające jeden, specyficzny błąd. Zadaniem
każdego testu jest sprawdzenie czy metoda rejestracji użytkownika wychwytuje
ten błąd i~czy zwraca odpowiedni wyjątek.

Niżej jest opisany przykładowy test posiadający błąd w~danych dotyczących karty
kredytowej klienta.

\begin{description}
  \item[Opis testu] \hfill \\
    Test sprawdza czy funkcja tworząca obiekt \f{Klient} odpowiednio reaguje na
    błędne dane związane z~podanymi przez niego informacjami o~karcie kredytowej.
  \item[Dane wejściowe] \hfill \\
    Niepoprawne, fikcyjne dane klienta.
  \item[Kryteria oceny] \hfill \\
    Metoda wyrzuca wyjątek.\\
    Typ wyjątku to \f{InvalidCreditCardInfoException}.
\end{description}

\subsection{Moduł aukcji}

\subsubsection{Wystawianie przedmiotu na aukcję}
\begin{description}
  \item[Opis testu] \hfill \\
    Test sprawdza funkcję tworzącą aukcję na podstawie przekazanego jej obiektu
    przedmiotu.
  \item[Przygotowanie do testu] \hfill \\
    Aktywne, fikcyjne konto klienta.\\
    Fikcyjne przedmioty przypisane do tego konta.
  \item[Dane wejściowe] \hfill \\
    Obiekt reprezentujący przedmiot, który należy do klienta wywołującego
    testowaną funkcję.\\
    Cena wywoławcza, data rozpoczęcia aukcji.
  \item[Kryteria oceny] \hfill \\
    Zostaje utworzona aukcja, której właściciel, przedmiot oraz reszta danych
    są zgodne z~danymi wejściowymi.
\end{description}

\section{Testy wydajnościowe}

\subsection{Rejestracja użytkownika}
\begin{description}
  \item[Opis testu] \hfill \\
    Wygenerowane zostanie 50~tys. poprawnych, losowych rekordów z~danymi
    klientów. Zostaną one wprowadzone do bazy danych z~wykorzystaniem \f{API}
    systemu i~porównane z~wygenerowanym zestawem.
  \item[Dane wejściowe] \hfill \\
    Losowe dane klientów, generowane na podstawie danych z~książek
    telefonicznych.
  \item[Kryteria oceny] \hfill \\
    100\% zgodność danych wejściowych z~danymi w~bazie danych.\\
    Czas wprowadzania danych do bazy krótszy niż $0.3 s$.
\end{description}

\subsection{Wyszukiwanie aukcji}
\begin{description}
  \item[Opis testu] \hfill \\
    Zostanie automatycznie przeprowadzonych tysiąc wyszukiwań na grupie
    wcześniej przygotowanych stron aukcyjnych. Wyszukiwania będą oparte na
    liście tagów. Tagi wprowadzane będą do wyszukiwarki systemu aukcyjnego
    dostępnej użytkownikom z~interfejsu webowego.
  \item[Przygotowanie do testu] \hfill \\
    Wygenerowanie fikcyjnych, otagowanych stron aukcyjnych.
  \item[Dane wejściowe] \hfill \\
    Lista słów kluczowych występujących (lub nie) na stronach aukcyjnych.
  \item[Kryteria oceny] \hfill \\
    Średni czas jednego wyszukiwania poniżej $10 ms$.\\
    Odpowiedni podzespół przygotowanych stron aukcyjnych musi być
    przedstawiony, w~zależności od wyszukiwanego słowa kluczowego.
\end{description}

\section{Testy akceptacyjne}

\subsection{Tworzenie aukcji}
\begin{description}
  \item[Opis testu] \hfill \\
    Tester stworzy grupę aukcji i spróbuje je uruchomić.
  \item[Dane wejściowe] \hfill \\
    Nazwy oraz treść stron aukcyjnych.
  \item[Kryteria oceny] \hfill \\
    Poprawność danych wejściowych z zapisanymi w~bazie i~wyświetlanymi na
    stronach.\\
    Prawidłowa reakcja na próby stworzenia duplikatów istniejących stron.\\
    Notyfikacje dostarczone do opiekunów aukcji.
\end{description}

\subsection{Wystawianie przedmiotu na aukcji}
\begin{description}
  \item[Opis testu] \hfill \\
    Testerzy przejdą przez cały proces wystawiania przedmiotu na aukcji.
    Zaczynając od wypełnienia formularza i przekazania przedmiotu kurierowi,
    do pojawienia się strony aukcyjnej w serwisie.
  \item[Dane wejściowe] \hfill \\
    Opis przedmiotu biorącego udział w teście.
  \item[Kryteria oceny] \hfill \\
    Średni czas transportu przedmiotu oraz katalogowanie w magazynie domu
    aukcyjnego po zgłoszeniu w systemie nie przekracza 5 dni roboczych.\\
    Odpowiednia reakcja systemu w~razie napotkania nieścisłości w formularzu
    wysyłkowym klienta.\\
    Porównanie oczekiwań klienta z stworzoną stroną aukcyjną.
\end{description}

\subsection{Udział w~aukcji}
\begin{description}
  \item[Opis testu] \hfill \\
    Test ma zweryfikować czy udział w~aukcji jest zrozumiały, wygodny oraz
    odpowiednio responsywny do przeprowadzania aukcji w~czasie rzeczywistym.
    Testerzy używają systemu, wypełnionego fikcyjnymi aukcjami w~różnych
    stanach i~biorą udział w~aktywnych.
  \item[Przygotowanie do testu] \hfill \\
    Wygenerowanie fikcyjnych aukcji.
  \item[Kryteria oceny] \hfill \\
    Wszystkie powiadomienia o~aukcjach są dostarczane tj. do właściciela
    przedmiotu, do kupującego, do pracowników, do magazynu.\\
    Klienci są w~stanie brać udział w~aukcjach i~nie uważają procesu za
    trudny.\\
    Przebieg aukcji i~jej strona są odświeżane automatycznie bez, lub
    z~opóźnieniem na tyle małym, że nie wpływa bezpośrednio na przebieg aukcji.
\end{description}

\end{document}

